\section{Usability}
\label{sec:usability}

Usability is one of the attributes of sustainable software, allowing users to achieve their “goals with effectiveness, efficiency, and satisfaction” \cite[p.3]{Venters_WSSSPE}. However, usability has the reputation of being a neglected aspect of scientific software \cite{Ahmed:2014}, and of relative perceived importance to users and developers \cite{Nguyen-Hoan:2010, Hucka:2016}. Papers on best practices for scientific software often omit the subject \cite{Stodden_WSSSPE, Wilson:2016} or briefly mention a particularity of scientific software usability, such as the convenience of command-line interfaces \cite{bestprSC}. Nevertheless, there is a significant number of informative case studies and guidelines on the subject \cite{Springmeyer:1993, Pancake:1996, Javahery:2004, Schraefel:2004,Letondal:2004, Talbott:2005, Macaulay:2009, DeRoure:2009, Keefe:2010, DeMatos:2013, Ahmed:2014}.

Scientific software usage and development present many challenges for usability design. Those can be related to development models, user-base needs and specialization, professional practices, technical constraints, and scientific demands \cite{Queiroz:2016}. As a result, computational science presents itself as an idiosyncratic field with unique and, occasionally, counterintuitive usability requirements. Throughout the next subsections, supported by documented case studies and further literature on the subject, we discuss those needs and how to address them. 

\subsection{Best practices recommendations}

\subsubsection{Observe how users work}

A recurring advice in most guidelines and case studies is to develop a firm understanding on how scientists do their work. A possible path to such understanding is the Designer as Apprentice technique, in which designers are tutored by users on the meanderings of their scientific work \cite{Springmeyer:1993}.  It is also important to analyse that work within the environment where it actually takes place \cite{Pancake:1996} and also evaluate how users work with similar tools  \cite{Javahery:2004}, if that is the case. 

% user-centered design

% specific communities / users
Besides designing for an actual user-base, it is advisable that scientific domain experts are brought into the design process for tool evaluation \cite{ Keefe:2010} and domain best practices \cite{Schraefel:2004,  DeMatos:2013}.

% specific needs 
Computational scientists often need specific features such as  

\subsubsection{Make small, incremental changes}

\subsubsection{Devise ways of entering and reading large amounts of data}

\subsubsection{Offer alternatives to graphical user interfaces}

\subsubsection{Allow users to participate and contribute} Make it extensible

\subsubsection{Keep UI code separate from scientific simulations}

\subsubsection{Be minimalistic, but look out for special needs}

\subsubsection{Design for insight}

\subsubsection{Offer ways of navigating through previous commands}

