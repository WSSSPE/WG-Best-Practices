\section{Usability}
\label{sec:usability}

Usability is one of the attributes of sustainable software, allowing users to achieve their “goals with effectiveness, efficiency, and satisfaction” \cite[p.3]{Venters_WSSSPE}. However, usability has the reputation of being a neglected aspect of scientific software \cite{Ahmed:2014}, and of relative perceived importance to users and developers \cite{Nguyen-Hoan:2010, Hucka:2016}. Papers on best practices for scientific software often omit the subject \cite{Stodden_WSSSPE, Wilson:2016} or briefly mention a particularity of scientific software usability, such as the convenience of command-line interfaces \cite{bestprSC}. 

Scientific software usage and development present many challenges for usability design. Those can be related to development models, user-base needs and specialization, professional practices, technical constraints, and scientific demands \cite{Queiroz:2016}. As a result, computational science presents itself as an idiosyncratic field with unique and, occasionally, counterintuitive usability requirements. Throughout the next subsections, supported by documented case studies and further literature on the subject, we discuss those needs and how to address them. 

\subsection{Best practices recommendations}

\subsubsection{Observe how users work (and start from there)}

\subsubsection{Offer alternatives to graphical user interfaces}

\subsubsection{Devise ways of entering and reading large amounts of data}

\subsubsection{Make it extensible}

\subsubsection{Keep UI code separate from scientific simulations}

\subsubsection{Be minimalistic, but look out for special needs}

\subsubsection{Design for insight}

\subsubsection{Offer ways of navigating through previous commands}

